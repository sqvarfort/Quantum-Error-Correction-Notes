\documentclass[12pt,a4paper]{article}

\setlength{\parindent}{0.1 in}
%\setlength{\parskip}{0.1 in}
\setlength{\oddsidemargin}{0.25 in}
\setlength{\evensidemargin}{-0.25 in}
\setlength{\topmargin}{-0.5 in}
\setlength{\textwidth}{7.0 in}
\setlength{\textheight}{9.5 in}
\setlength{\headsep}{0.45 in}

%\usepackage[fleqn]{amsmath}
%\usepackage{amsfonts,graphicx}
\usepackage{amsmath,amsfonts,graphicx}
\usepackage[fleqn]{mathtools}
\usepackage{setspace}
\usepackage{hyperref}
\usepackage[nottoc]{tocbibind}
\usepackage{tocloft}
\usepackage[outermargin=2 in]{geometry}
\usepackage{scrextend}
\usepackage{tensor}
\usepackage{cancel}
\usepackage{slashed}


%Adding `Appendix' to the appendices
\usepackage[toc,page]{appendix}

%Add a bullet point to description items
\usepackage{enumitem}

%Mathematics
\usepackage{braket}
\usepackage{ulem}
\usepackage{xcolor}
\usepackage[font={small,it}]{caption}

\bibliographystyle{unsrt}

%New commands
%Maths
\newcommand{\beq}{\begin{equation}}
\newcommand{\eeq}{\end{equation}}
\newcommand{\bea}{\begin{align}}
\newcommand{\eea}{\end{align}}
\newcommand{\p}{\partial}
\newcommand{\trace}[1]{\mathrm{Tr}\left[#1 \right]}
\newcommand{\ptrace}[2]{\mathrm{Tr}_{#1} \left[ #2 \right]}
\newcommand{\bpmat}{\begin{pmatrix}}
\newcommand{\epmat}{\end{pmatrix}}
\newcommand{\vv}[1]{\vec{#1}}
\newcommand{\mat}[1]{\uuline{#1}}
\newcommand{\norm}[1]{\| #1 \|}
\newcommand{\op}[1]{\mathbb{#1}}
\newcommand{\vhat}[1]{\hat{\vv{#1}}}

%Renewed commands, in order for them to take arguments with automatically adjusted brackets 
\renewcommand{\dim}[1]{\mathrm{dim}\left( #1\right)}
\renewcommand{\det}[1]{\mathrm{det} \left( #1 \right)}
\renewcommand{\exp}[1] {\mathrm{exp} \left[ #1 \right]}


%\mathbb Letters
\newcommand{\identity}{\mathbb{I}}
\newcommand{\inreal}{\mathbb{R}}
\newcommand{\incomplex}{\mathbb{C}}

%Redefine Braket
\renewcommand{\braket}[1]{\left\langle #1 \right\rangle}

%Integration
\newcommand{\intd} {\mathrm{d}}

%Operators
\newcommand{\phihat}{\hat{\phi}}
\newcommand{\xhat}{\hat{x}}
\newcommand{\phat}{\hat{p}}
\newcommand{\Dhat}{\hat{D}}
\newcommand{\Hhat}{\hat{H}}
\newcommand{\ahat}{\hat{a}}
\newcommand{\bhat}{\hat{b}}
\newcommand{\chat}{\hat{c}}
\newcommand{\Phihat}{\hat{\Phi}}

%Channels
\newcommand{\channel}[3]{\mathcal{#1}^{#2 \rightarrow #3}}


%Caligraphy letters
\newcommand{\cl}[1]{\mathcal{#1}}
\newcommand{\Hilbert}{\mathcal{H}}
\newcommand{\calN}{\mathcal{N}}
\newcommand{\Lag}{\mathcal{L}}
\newcommand{\calD}{\mathcal{D}}


%Pauli
\newcommand{\Xhat}{\hat{X}}
\newcommand{\Yhat}{\hat{Y}}
\newcommand{\Zhat}{\hat{Z}}
\newcommand{\PauliX}{\bpmat 0 & 1 \\ 1 & 0 \epmat}
\newcommand{\PauliZ} {\bpmat 1 & 0 \\ 0 & -1\epmat}



%Gell-Mann matrices
\newcommand{\GMone} {\bpmat 0 & 1 & 0 \\ 1 & 0 & 0 \\ 0 & 0 & 0 \epmat }
\newcommand{\GMsix}{\bpmat 0 & 0 & 0 \\ 0 & 0 & 1\\ 0 & 1 & 0\epmat}

%Density matrices
\newcommand{\rhotwo}{\bpmat 1 & e^{-it} \\ e^{it} & 1 \epmat}
\newcommand{\rhothree} {\bpmat 1 & e^{it} & e^{2it} \\
e^{-it} & 1 & e^{it} \\
e^{-2it} & e^{-it} & 1 \epmat}



%Misc
\def\dbar{{\mathchar'26\mkern-12mu d}} %a $d$ with a bar through its stem


\newcommand{\eq}[1]{$#1$}

%Undertilded quantities
\newcommand{\tildeq}{\underset{^\sim}q}
\newcommand{\tildep}{\underset{^\sim}p}

%Curly letters
\newcommand{\calE}{\mathcal{E}}

\newcommand{\Nhat}{\hat{N}}

%Wave vector shortening
\newcommand{\kvec}{\vv{k}}












\begin{document}
\title{Error Correction Lecture 4}
\author{with Dan Browne}
\maketitle
\tableofcontents
\section{The Stabiliser Formalism}
This is a state-independent formalism. It was first introduced by Gottesmann, This is by far one of the most important formalisms that made error correction a realistic prospect. 

The Stabiliser formalism can correct any Pauli Error. It relies on the mathematical properties of Pauli matrices, and can also be used to explain multi-qubit entanglement. It is a very useful tehnique. 

\subsection{Example: 3-qubit repetition code}

The 3-qubit repetition code is an example of the stabiliser formalism. For an arbitrary state
\beq
\ket{\psi} = \alpha \ket{000} + \beta \ket{111}
\eeq
we can detect a $X$ bit flip error by measuring $ZZI, ZIZ$ or $IZZ$. Because the third measurement is superfluous, we need only measure the first two. 

Recall the effect of an error on the qubits. We have

\begin{tabular} {cc}
Error & States \\ \hline
$XII$ & $\alpha \ket{100} + \beta\ket{011}$ \\
$IXI$ & $\alpha \ket{010} + \beta \ket{101}$ \\
$IIX$ & $\alpha \ket{001} + \beta \ket{110}$
\end{tabular}

Similarly, for a syndrome measurement using the previous measurements, we find

\begin{tabular}{ccc}
$ZZI$ & $ZIZ$ & $IZZ$ \\ \hline
+ & + & + \\
-- & -- & + \\
-- & + & -- \\
+ & -- & --
\end{tabular}

From this, we note that we are clearly dealing with a non-degenerate code. But instead of looking at the state and working out the entire syndrome table, we can ask: does the error commute with the error detection? That is, 
\beq
[M, E] =^? 0
\eeq
We can summarise this in a table as well. 

\begin{tabular}{c|ccc}
& $ZZI$  & $ZIZ$ & $IZZ$ \\ \hline
$III$ & $\surd$ & $\surd$ & $\surd$ \\
$XII$ & $\times$ & $\times $ & $\surd$ \\
$IXI$ & $\times $ & $\surd$ & $\times$ \\
$IIX$ & $\surd$ & $\times$ & $\times$ 
\end{tabular}
Note the similarity of the tables. 

We can then prove that given an error $E$ and error detection measurement $M$, if they anti-commute, we get $-1$ and if they commute, we get $+1$, should we measure. 

So when $M$ anti-commutes with the error $E$, we detect the error. Note that all the error detecting measurements commute. And note as well that on an error-free codewords, all outcomes of $M$ are + (which is exactly what we want) because
\beq
[M, I] = 0
\eeq
NOte talso that there is a product sturcture outcome of $IZZ$, since it is a product of outcomes $ZIZ$ and $ZZI$. That is, we still only need to make two measurements because even the outcome of the third one will always be obtained from the other two. 

These important group properties will come into play here. But before we proceed, we must show that the Pauli group is closed. THe identity, inverse and associativity are already taken care of. 

\section{Definition of the Stabiliser Group}
We must first define the notion of \textbf{stabilisation}. A group element is stabilised if another group element leaves it invariant. 

Thus, given a set of codewords space basis states, $\ket{x}_L$, where for $x = 00\ldots 00$ to $x = 11 \ldots 11$ (which we can call binary vectors) we say that $P$ stabilises the odespace if
\beq
P\ket{x}_L = \ket{x}_L
\eeq
for all $x$. Note that this is an eigenvalue equation with eigenvalue $+1$. Thus, $\ket{x}_L$ is an eigenvector of $P$ with eigenvalue  $+1$ for all $x$. 

In the Stabiliser formalism, we identify Pauli operators that we measure to detect errors with operators that stabilise our codewords. We saw the stabiliser for $\ket{000}$ and $\ket{111}$. They are $ZZI$, and $ZIZ$. What properties will they have? 

It is not always true that anti-commuting operators do not share eigenvalues. Two Pauli operators share a joint eigenbasis if and only if they commute. 

Consider therefore two commuting Pauli operators $P$ and $Q$ that both stabilise $\ket{x}_L$. We find that 
\beq
P\ket{x}_L = \ket{x}_L
\eeq
\beq
Q \ket{x}_L = \ket{x}_L
\eeq
Then, it follows that 
\beq
PQ \ket{x}_L = P\ket{x}_L = \ket{x}_L
\eeq
So, $PQ$ does also stabilise $\ket{x}_L$. 

Thus, given a code space $\ket{x}_L$, its stabiliser group is the set of all Pauli operators $S_j$ that satisfy
\beq
S_j \ket{x}_L = \ket{x}_L
\eeq
for all $x$ and for all $j$. The set $\{S_j\}$ forms a group! It is a subgroup of the $N$-qubit Pauli group $\{i, X, Z\}$. 

Recall that a group with only commuting elements is abelian. The stabiliser group only has commuting operators, and hence is an abelian subgroup of the Pauli group. 

Any quantum error correcting code for which the error detection measurements are an abelian subgroup of the $N$-qubit Pauli group is called a stabiliser group. 

All codes seen in the course so far are in fact examples of stabiliser groups. 

\textbf{Exercise for the student}: Go back and check this. Look at the commutators of the various operators. 

For example, the 3-qubit code has stabilisers $ZZI, ZIZ, IZZ, III$. The 3-Qubit phase flip code has stabilisers $XXI, XIX, IIX, III$. Fruthermore, the Shor and the Steane code and the 5-qubit code are all examples of stabiliser codes. Finally, we have topological surface codes, which also fall into this category. 

Note that nay group can be represented by its generators. Thus, for our future studies of stabiliser groups, we will have to look only at the stabiliser generators. 

\section{Order of group and group generators}
The \textbf{order} of the group is the number of group elements. 

Any group $G$ can be described a set of generators (which is not unique) from which all elements can be obtained. We write this as
\beq
G = \braket{g_1 \ldots g_n}
\eeq

As an example, take the 3-qubit code. For the 3-qubit code, we have three possible sets of generators for the stabiliser group. $\braket{ZZI, ZIZ}$, $\braket{ZIZ, IZZ}$ and $\braket{ZZI, IZZ}$. 

In general, a stabiliser group with $m$ generators has $2^m$ elements. The generators must be \textbf{independent}. By independent, we mean that we choose the minimum number of generators that can generate the entire group. This is similar to choosing a vector basis. 

For self-inverse elements, such as the Pauli group, $2^m$ also holds. Since $(g_j)^2 = I$ and since elements commute, the order of multiplication does not matter. Any element can in fact be written
\beq
S = \prod_{j = 1}^m g_j^{x_j}
\eeq
where $x_j$ is an $m$-bit string. Then, we do indeed get $2^m$ elements. 

So, $m$ is always our generator index. In fact, describing the group by its generators gives us an exponential saving, since in order to speak of the group,  we only have to consider $m$ elements instead of $2^m$ elements. 

However, how many generators do we need for the stabiliser formalism? 

\section{The Number of Generators}
For the 3-qubit code, we had $m = 2$, $n = 3$ and $k = 1$. For the Shor code, we similarly had $n = 9$ and $k = 1$. The generators for that stabiliser group were $Z_1Z_2, Z_2Z_3, Z_3Z_4, Z_4Z_5, Z_5Z_6, Z_6Z_7, Z_7Z_8Z_8Z_9$ and $X_1X_2X_3X_4X_5X_6, X_1X_2X_3X_7X_8X_9$. 

From these two examples, we can see a pattern. In general
\beq
m = n-k
\eeq
This is satisfied by any stabiliser code. In words, given $k$ qubits encoded in $n$ physical qubits, the stabiliser has $m$ independent generators. A proof can e found in one of the problem sheets. 

The key idea is to use $m$ generators to describe the entire code. 

\section{Aside:Stabiliser states}
If $k = 0$ we end up with $m = n$. That is, we have no encoded states at all. We say that we have a 0-dimensional code space, which means that there is one single vector in the codespace. Note that they do not encode an entire qubit, since this requires two basis states. Instead, this code can be used to encode one single state. 

States like these are called \textbf{stabiliser states}. It is a state which is the joint $+1$ eigenstate of $n$ independent, commuting pauli operators (since $m = n$). A $k = 0$ code is also called a stabiliser code. 

They are important because they use canonically entangled states. 

Here are some examples of single states and their stabiliser generators. 

\begin{tabular}{cc}
State $(n = 1)$& Stabiliser generator \\ \hline
$\ket{0}$ & $Z$ \\
$\ket{1}$ & $-Z$ \\
$\ket{+}$ & $X$ \\
$\ket{-}$ & $-X$ \\
$\ket{+i}$ & $Y$ \\
$\ket{-i}$ & $-Y$
\end{tabular}

And for states with more than one physical qubit,
 
\begin{tabular}{ccc}
States $n = 2$ & Stabiliser generator  & Full group\\ \hline
$\ket{00} + \ket{11}$ & $ZZ, XX$  & $ZZ, XX, -YY, II$\\
$\ket{00} - \ket{11}$ & $ZZ, -XX$  & $ZZ, -XX, YY, II$\\
$\ket{01} + \ket{10}$ & $-ZZ, XX$ & $-ZZ, XX, YY, II$\\
$\ket{01} - \ket{10}$ & $- ZZ, -XX$ & $-ZZ, -XX, -YY, II$
\end{tabular}

For $n = 3$ physical qubits, we end up with states such as the GHZ state: $\ket{000} + \ket{111}$ which is stabilised by $XXX; ZZI, ZIZ$. For any $n >3$, we get so-called cluster states or graph states. 

\section{Detection of Errors in the Stabiliser Formalism}
Consider a stabiliser codeword $\ket{\psi}$ and its stabiliser element $S_j$. Let this codeword be affected by a Pauli error $E$. The final state is $E\ket{\psi}$ and we measure $S_j$. Then, if
\beq
[S_j, E] = 0
\eeq
the outcome is $+1$. However, if 
\beq
\{S_j, E\} = 0
\eeq
the outcome is $-1$ and we know that an error has occurred. 

Proof: We want to know what the eigenstates of $E$ applied to the state is. That is, what are the eigenstates $\lambda_j$ in 
\beq
S_j \left( E\ket{\psi}\right) = \lambda_j \left(E \ket{\psi}\right)
\eeq
Assume first that $[S_j, E] = 0$. Then, 
\beq
S_j E \ket{\psi} = E S_j \ket{\psi} = E\ket{\psi}
\eeq
From which we see that the eigenvalue is indeed $+1$. However, if $\{S_j, E\} = 0$, we have
\beq
S_j E \ket{\psi} = - E S_j \ket{\psi} = - E \ket{\psi}
\eeq
The eigenvalue is $-1$. 

The central question in the stabiliser formalism is: does the error commute or anti-commute with the stabiliser? 

\section{Limitations of a Stabiliser Code}
For a Pauli error $E$ to be detectable, it must anti-commute with at least 1 element of the stabiliser. So, must we then go on to consider all elements in the group? It turns out that we can map the entire behaviour of the code by just considering the generators. If the error anti-commutes with just a single generator, it is detectable. This works because $S_j$ is a product of generators. 

So the only thing we need to measure are the generators. However, any error that commutes with all generators is undetectable. E.g. $Z$ in the bit-flip code, which we already saw was undetectable. Recall that the generators were $[XXX, ZIZ, IZZ]$ and so a single $Z$ error would indeed commute with them. 

Question: It seems to me that $Z$ does not commute with $XXX$. However, $XXX$ is clearly a generator, as it maps the codeword $\ket{000} + \ket{111}$ into itself. 

Attempt at answer: $XXX$ only maps one single codeword onto itself, namely the superposition. For a state in $\ket{000}$, it does not map onto the same codeword. However, does this means that it is not a generator? 

To clarify, 
\beq
XXX \left( \alpha \ket{000} + \beta\ket{111} \right) = \alpha \ket{111} + \beta \ket{000}
\eeq
Here, $XXX$ is a logical $X$ operator which maps codespace vectors back into the codespace. If an error is undetectable, it must be a logical operator on the codespace, and vice versa. In particular in stabiliser codes, undetectable Pauli errors are logical Pauli operators. 

So, the logical Pauli operators are the set of logical operators that commute with the entire stabiliser. 

Again, we can look at the bit flip code. The stabiliser itself is $ZZi, IZZ, ZIZ, III$. All of these are equal to the logical identity operator $\bar{I}$. These are \emph{not} errors (but they could be, and then we couldn't possibly correct for them). 

It turns out that every element of the stabiliser is always $\bar{I}$. But, note that for $XXX = \bar{X}$, we have
\beq
(XXX) S_j \ket{\psi} = (XXX) \ket{\psi}
\eeq
So any product of the stabiliser and $\bar{X}$ is also a logical operator! So, given out three generators, we have four equivalent $\bar{X}$, 
\beq
XXX, -YYX, -YXY, -XXY
\eeq
The same goes for $\bar{Z}$ and $\bar{Y}$. This is a way to find all the undetectable errors. Recall that for the logical $\bar{Z}$ for the bit flip code, we have $\bar{Z} = ZII$. So we get
\beq
IZI, IIZ, ZZZ
\eeq
are all equivalent $\bar{Z}$. 

These are examples of \textbf{cosets}. In fact, we say that these are constructions of cosets. We can easily identify these sets of operators. This works for all stabiliser codes. We have that 
\beq
\bar{Y} = \mbox{any logical } \bar{X} \times  \mbox{any logical } \bar{Z}
\eeq
We can prove this by writing
\beq
\bar{X} S_j \bar{Z} S_k = \bar{X} \bar{Z} (S_j S_k)
\eeq
This creates four different logical $\bar{Y}$. 

\section{Overview of Error Correction}
Let us here summarise what we have learnt so far. 

\begin{description}
\item[Detection] An error $E$ is detected if it anti-commutes with at least one stabiliser generator. 

\item[Correction] From the syndrome, we can determine a correction operator $C$. Then we apply  $C$ to the state. 
 
\item[Stabilisation] It turns out that if $C$ and $E$ both form the stabiliser, such that 
\beq
CE \ket{\psi} = S_j  \ket{\psi}
\eeq
then the error is successfully corrected. This way, we can easily find the right correction method. Otherwise, $CE$ must be a logical operator. Note that error correction can always fail!

\end{description}

\subsection{Successful correction example}
Consider the error $XII$. If we apply the generators, we see that $ZZI$ anti-commutes with the error and $IZZ$ commutes. Thus, we get a $-1$ outcome for the first one and a $+1$ for the second one. 

Then, we correct the error with $XII$. We see that
\beq
(XII)^2 = III
\eeq
This is a stabiliser, and so the error is corrected. 

Consider now the error $IXX$. It has syndrome $-1$ for $ZZI$ and $+1$ for $IZZ$. We cannot correct this error because its syndrome is degenerate with the first case. Applying the same correction method will give us $XXX = \bar{X}$. 

However, since 1 error is more likely than two, out chances are pretty good. 

\section{How to compute codeword kets}
Let us now see how we recover the states from the Stabiliser formalism. That is, given a stabiliser formalism, what are the codewords? From the stabiliser generators and logical operators, we can find the states. 

We first make an observation. Let $\braket{g_1 \ldots g_n}$ be the stabiliser generators, and let $\bar{Z}$ be the logical $Z$ operator. Then, $\ket{0}_L$ states obey
\beq
S_j \ket{0}_L = \ket{0}_L
\eeq
\beq
\bar{Z} \ket{0}_L = \ket{0}_L
\eeq
In fact $\ket{0}_L$ is the only state that satisfies both of these equations. However, this is cumbersome. Instead, we can create projectors onto this codespace. This leaves behind the codespacae vector. 

So, for each $g_j$, we construct a projector
\beq
P_j = \frac{I + g_j}{2}
\eeq
We can check that it is a projector, 
\beq
P^2 = P = \left( \frac{I + g_j}{2} \right)^2 = \frac{I + g_j }{2}
\eeq
Then, for any element in the codespace, we require
\beq
\left( \frac{i + g_j}{2} \right) \ket{x} = \ket{x}
\eeq
Anything outside the codespace (such as an error) will give
\beq
g_j \ket{\psi} = - \ket{\psi}
\eeq
We can rearrange the above to get
\beq
\left( \frac{I+ g_j}{2} \right) \ket{\psi} = 0
\eeq
Form this we see that the projector indeed does only project onto the codespace. WE are essentially removing anything orthogonal to the codespace. We can take the product of all projectors, 
\beq
\prod_{j = 1}^m \left( \frac{I + g_j}{2}\right) = P
\eeq
This is the projector onto the entire codespace. Every vector in the codespace is mapped to the codespace. 

Now, let the projector be 
\beq
\frac{I + \bar{Z}}{2}
\eeq
It has the following properties. It leaves $\ket{0}_L$ unchanged, But, we find
\beq
\frac{I + \bar{Z}}{2} \ket{1}_L = 0
\eeq
Then we can show that 
\beq
\prod^m_{j = 1}\left( \frac{I + g_j}{2} \right) \left( \frac{I + \bar{Z}}{2} \right) = \ket{0}_L \bra{0}_L
\eeq
because the $\bar{Z}$ projector picks out the same projector from $P$. Thus, we can easily obtain the codewords if we know the logical operators and the generators. 

Another property is that 
\beq
\prod_{j = 1}^m \left( \frac{I + g_j}{2} \right) = \frac{1}{2^m} \sum_{ \forall S \in \mbox{Stabiliser group} }S
\eeq

\section{5-qubit code}
Let us now have a look at the 5-qubit code. It is defied within the Stabiliser formalism via its operators. It is a $(n = 5, k= 1, m = 4)$ code. We have
\beq
g_1 = XZZXI
\eeq
\beq
g_2 = IXZZX
\eeq
\beq
g_3 = XIXZZ
\eeq
\beq
g_4 = ZXIXZ
\eeq
The logical operators are (simply enough)
\beq
\bar{Z} = ZZZZZ
\eeq
\beq
\bar{X} = XXXXX
\eeq
From the stabilisers and the logical, we can work out the weight for the minimum undetectable error. We ask: What is the smallest weight logical operator? 

We can calculate the distance of the code. Consider $\bar{Z}g_1$, which is another logical $Z$. We find
\beq
\bar{Z} g_1 = - YIIYZ
\eeq
This is weight 3 (count all the operators that are not identity). This is the smallest weight undetectable error. Hence, the code distance is $d = 3$. The number of arbitrary errors that can be detected are
\beq
\frac{3 - 1}{2} \sim 1
\eeq
Note that this does \emph{as well} as the Steane code. See NC for the entire list of the codewords, section 10.104. They are superpositions of 16 terms, so we will not write them out here! 

 


\end{document}